% This file was converted from HTML to LaTeX with Nathan Torkington's
% html2latex program
% Version 0.9c
\documentstyle{article}
\begin{document}
\section*{NAME}
html2latex -- convert HTML markup to LaTeX markup
\section*{SYNOPSIS}{\tt html2latex {\it [opt ...] [file ...]}}\section*{DESCRIPTION}
For each file argument, {\it html2latex} converts the text as
HTML markup to LaTeX markup.  If no files are specified, a usage
message is given.  Input will be taken from standard input for files
named {\it -}.  Output will to a similarly named file with a
{\bf .tex} extension ({\it html2latex} recognises
{\bf .html} extensions).
\par 
Options modify the action of {\it html2latex}.  The options are:
\begin{description}\item[-n]Number sections.
\item[-p]Place page breaks after the title page (if present) and the
table of contents (if present).
\item[-c]Generate a table of contents.
\item[-s]Create no files -- LaTeX is output to stdout.
\item[-t Title]Generate a title page, with the title ``Title''.
\item[-a Author]Generate a title page, with the author ``Author''.
\item[-h Header]Place the text ``Header'' after $\backslash$begin\{document\}.
\item[-f Footer]Place the text ``Footer'' before $\backslash$end\{document\}.
\item[-o Options]Specify the options to $\backslash$documentstyle.
\end{description}\section*{EXAMPLES}
An example of use is
\begin{verbatim}html2latex -n - < file.html | less
\end{verbatim}
This converts {\bf file.html} to LaTeX and pages through the
output.  The sections (corresponding to heading tags in the HTML
source) will be numbered.
\par 
Another example is
\begin{verbatim}html2latex -t 'Introduction to HTML' -a gnat -p -c -o
'[bookman]{article}' html-intro
\end{verbatim}
This takes input from the file {\bf html-intro}, writing to
{\bf html-intro.tex}, and adds a title page (with title
{\it Introduction to HTML} and author {\it gnat})
and table of contents with page-breaks after both.  The sections of
the document are not numbered.  The LaTeX source includes the line
$\backslash$documentstyle[bookman]\{article\}.
\section*{SEE ALSO}
latex(1)
\section*{BUGS}
Current the only HTML tags supported are: {\bf TITLE, H1, H2, H3, H4, H5,
H6, UL, OL, DL, DT, DD, LI, B, I, U, EM, STRONG, CODE, SAMP, KBD, VAR,
DFN, CITE, LISTING}.  The only recognised SGML escapes are {\bf \&.amp,
\&.lt, \&.gt}.  {\bf ADDRESS} tags are handled badly.
\par 
The {\bf COMPACT} attribute to a {\bf DL} tag is not recognised.
{\bf MENU} and {\bf DIR} styles are not handled well.
{\bf TITLE} text are ignored.
\par 
Currently {\bf PRE} tags are not handled at all.
\par 
The entire file is read into memory.  For long HTML documents on
machines with little memory, this may cause problems.
\section*{CREDITS}
Nathan Torkington adapted the HTML parser from NCSA's Xmosaic package
({\bf file://ncsa.uiuc.edu/Web/xmosaic}) and wrote the conversion
code.  The HTML parser code is subject to the NCSA restrictions.  The
conversion code is subject to the VUW restrictions.  Enquiries should
be sent via e-mail to {\tt Nathan.Torkington@vuw.ac.nz}.\end{document}
